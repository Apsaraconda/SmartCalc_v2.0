\documentclass{report}
% подключаем русский шрифт
\usepackage[utf8]{inputenc}
\usepackage[russian]{babel}
\righthyphenmin=2
\voffset = 0pt
\topmargin = 0pt
\headsep = 0pt
\headheight = 0pt
\textheight = 609pt
\oddsidemargin = 31pt
\marginparsep = 0pt
\marginparwidth = 0pt
\textwidth = 360pt

% начинаем документ
\begin{document}
{\bfseries SmartCalc\_v2.0} - это комплексный инструмент для решения различных математических задач, выраженных в инфиксной нотации.

Программа реализована на графической библиотеке QT.

Калькулятор имеет возможность рассчитывать результат введенных выражений, включая тригонометрию и расчет корня числа.

Графический калькулятор позволяет отображать график функции от X.

\section* {\bfseries Установка.}

Чтобы начать использовать SmartCalc, его необходимо установить с помощью команды make install. Эта команда установит программу в выше расположенную папку SmartCalc\_v2.0

\section* {\bfseries Удаление.}

Удалить приложение можно с помощью команды make uninstall.

\section* {\bfseries ZIP.}

Архивировать проект можно с помощью команды make dist.

\section* {\bfseries Основные особенности:}
\begin{itemize}
\item Программа реализована с использованием паттерна MVC;

\item На вход программы могут подаваться как целые числа, так и вещественные числа, записанные и через точку, и в экспоненциальной форме записи вида XeY;

\item Вычисление производится после полного ввода вычисляемого выражения и нажатия на символ =;

\item Вычисление произвольных скобочных арифметических выражений в инфиксной нотации;

\item Вычисление произвольных скобочных арифметических выражений в инфиксной нотации с подстановкой значения переменной x в виде числа;

\item Построение графика функции, заданной с помощью выражения в инфиксной нотации с переменной x  (с координатными осями, отметкой используемого масштаба и сеткой с адаптивным шагом);

\item Возможно менять масштаб графика;

\item Область определения и область значения функций ограничиваются по крайней мере числами от -1000000 до 1000000;

\item Для построения графиков функции необходимо можно менять отображаемые область определения и область значения;

\item Точность дробной части - минимум 7 знаков после запятой;

\item У пользователя есть возможность ввода до 300 символов;

% Скобочные арифметические выражения в инфиксной нотации должны поддерживать следующие арифметические операции и математические функции:
\end{itemize}
\end{document}